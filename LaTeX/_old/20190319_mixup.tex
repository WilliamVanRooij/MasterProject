\section{Summary to make sure I understand what happens}
We want to approximate $p(z|x)$ where $z$ are the parameters, $x$ the observed values.

We want to to find $q(z)$ that minimizes $KL(q||p)$, hence maximizes $ELBO(q)$.

We consider $q = \prod q_j (z_j))$ (mean-field approx.) (parameters independent)

Depending on the starting point of the CAVI algorithm to calculate the $q_j(z_j)$, we may arrive at a optimum local and not global. So we vary the starting points in hope to reach the global. Then we perform a "sort of" BMA on the $K$ models we obtain.

We want to estimate:
$$
p(M_k|x) = \frac{p(x|M_k)p(M_k)}{\sum_j p(x|M_j)p(M_j)}
$$

So we have to find the $ELBO$ for each of the models $M_k$ which will give us $p(x|M_k)$. We can consider the $M_k$ to be equiprobable so all $p(M_k)$ are the same.